\def\year{2016}
%File: formatting-instruction.tex
\documentclass[letterpaper]{article}
\usepackage{aaai}
\usepackage{times}
\usepackage{helvet}
\usepackage{courier}
\usepackage{subfigure}
\usepackage{algorithm}
\usepackage{multirow}
\usepackage{multicol}
\usepackage{algpseudocode} %format of the algorithm
\usepackage{color}
\usepackage{url}
\usepackage{latexsym}
\usepackage{graphicx}
\usepackage{epstopdf}
\usepackage{epsfig}
\usepackage{amsmath, bm}
\usepackage{amsfonts}
\usepackage{enumitem}
\usepackage{ulem}
\normalem
\frenchspacing
\setlength{\pdfpagewidth}{8.5in}
\setlength{\pdfpageheight}{11in}
\pdfinfo{
/Title (Insert Your Title Here)
/Author (Put All Your Authors Here, Separated by Commas)}
\setcounter{secnumdepth}{0}  
 \begin{document}
% The file aaai.sty is the style file for AAAI Press 
% proceedings, working notes, and technical reports.
%
\title{PKUICST at TREC 2016 Real-Time Summarization Track: \\
Push Notifications and Email Digest
}
\author{Lili Yao \quad Chao Lv \quad Feifan Fan \quad Yansong Feng\footnote{Corresponding author} \quad Dongyan Zhao\\
\{yaolili, lvchao, fanff, fengyansong, zhaody\}@pku.edu.cn\\
\\
Institute of Computer Science and Technology\\
Peking University, Beijing 100871, China\\
}

\maketitle
\begin{abstract}
\begin{quote}
This paper describes our approaches for the TREC 2016 Real-Time Summarization track,
including push notifications scenario and email digest scenario.
In the push notifications scenario, we design a real-time system, which listens to the Twitter sample stream and makes the push decisions for the given topics.
Low coupling modules are utilized to obtain the timely, relevant and novel features.

%In the email digest scenario, we apply pseudo-relevance feedback using language model and
%similarly we adopt an adaptive dynamic query-biased filtering method to choose the novel representative tweets.

%Besides, the results of scenario periodic email digest can promote the performance of scenario push notifications since we %utilize shared global relevance threshold. 

%Experimental results show that our adaptive query-biased filtering methods achieve good performance with respect to ELG and nCG metrics for push notifications scenario.

%In addition, our systems for scenario periodic email digest also obtain convincing nDCG scores.
\end{quote}
\end{abstract}

\section{Introduction}
With the rapid development of the microblog, shch as Twitter and Weibo,
the information that microblog covered is rather numerous than expected.
To explore user's interests and boost recommendation performance in real-time environment,
TREC first introduced Tweet Timeline Generation (TTG) track in 2014\cite{lin2014overview} and developed it in 2015.
The Real-Time Summarization Track in TREC 2016 is a real-time summarization task broken down into two scenarios,
which is aiming to explore techniques for monitoring streams of social media posts with respect to users' interest profiles.
Different from the last year's Microblog track, it requires a real on-line decision,
which means participating systems need to decide whether or not push notification for a tweet before seeing the subsequent tweets.
And two scenarios are described as follow:

\begin{itemize}
\item \textbf{Scenario A (mobile notification):} Content that is identified as relevant and novel by a system based on the user's interest profile should be sent to the user in a timely fashion. 
\item \textbf{Scenario B (email digest):} Participating systems should identify tweets and aggregate them into an email digest. The email should be periodically sent to a user. Under that circumstances, users can read a longer story about the contents.
\end{itemize}

In the push notifications scenario, our system requires to ``listen'' to
the Twitter API\footnote{https://github.com/lintool/twitter-tools}
and make real-time push actions for each interest profile.
We design a on-line system which contains three modules:
Filter Module, Judge Module and Submit Module.
When a new tweet $D$ comes, we use Filter Module to remove it
if it has no overlap words with all the interest profiles.
In the Judge Module, we first estimate the relevance score between the tweet
and each interest profile using the normailzed negative KL-divegence distance.
A tuned relavance threshold $\alpha$ is utilized to judge whether $D$ and
the interest profile are relavant.
Meanwhile, we keep a push queue for each interest profile.
Then, for every relavant interest profile $Q$, we estimate the novel score by comparing $D$
with previous tweets in its push queue.
Similarly, we use the normailzed negative KL-divegence distance and a tuned
novel threshold $\beta$ to decide whether $D$ is a novel one for the interest profile $Q$.
Submit Module is used to submit passed tweets to the Evaluation Broker with the power to 
handle error code from remote server.
It can also store the push queue for each interest profile and help recover our system
if any crash happens.

In the scenario of email digest, similar with scenario A,
we firstly clean the raw tweets generated between evaluation period.
To better understand the user intent,
we utilize Google web resource as external evidences to expand original query.
Then, the language model framework is applyed to estimate the relevance between given interest profile with candidate tweets.
For each interest profile,
we rank the candidate tweets of every day by the relevance scores which adopt two different smooth methods. 
Once we obtain the ranked tweet list,
we calculate the novelty scores between the candidate tweet with each tweet that has been pushed previously,
the novelty threshold $\gamma$ is used to determine whether the candidate tweet is included in the email digest.
And there are two kinds of strategies to measure the novelty,
which will be explained in Section 4 in detail.



\section{Preliminaries}
In this section, we first introduce the preliminaries for tweets in both scenarios. Our system will continuously monitor the Twitter's live tweet sample stream using the official API. As soon as the system obtains the json data of tweets, the system will preprocess the tweet text and filter tweets without any keywords in each user's profile.

\subsection{Pre-processing}
The preprocessing we adopt on tweet stream is described as follows:
\begin{itemize}
\item \textbf{Non-English Filtering:} Tweets written in a language other than English would be judged as not relevant based on guidelines of Real-Time Summarization Track. Thus, we use the twitter's language detector to abandone the non-English tweets.
\item \textbf{Non-ASCII Words:} Removing all NON-ASCII characters from the tweets will also helps remove non-English tweets.
\item \textbf{Redundant Retweet Elimination:} All additional commentary in the tweets containing 'RT @' will be ignored. As the guidline mentioned, all retweets should be normalized to the underlying tweets.
\item \textbf{Porter Stemming and Stopword Filtering:} We remove all stopwords and stem the tweet text using the Natural Language Toolkit.
\end{itemize}

\subsection{Filtering}
In order to boost the speed of identifying possible relevant tweets for each user's interest profile $q$, we simply filter tweets that do not contain any keywords for each profile $q$, and the rest tweets are chosen as candidate tweet collection $\Gamma$ for $q$.

\begin{figure}[htbp]
\centering
{
	\epsfig{file=figures/scenarioA.pdf,width=0.45\textwidth}
}
\caption{Scenario A System Framework.}
\label{fig:Asys}
\end{figure}

\section{Scenario A: Push Notifications}
\subsection{System Overview}
As noted above, the scenario A push notifications on a mobile phone aims to push relevant and novel tweets to users and such tweets are triggered a relatively short time after the content is generated.
In this section, we mainly discuss the architecture of our system, which is shown in \ref{fig:Asys}.
From the figure, we can see that our system mainly contains two components:
\begin{itemize}
\item \textbf{Offline Module:} We utilize the external web resources to obtain context words about each interest profile. Using the Google Web Search API, we obtain top five relevant documents consisting of titles and snippets. 
Then we preprocess the documents and incorporate with original query to generate a merged query document which has a more comprehensive word distribution.
\item \textbf{Online Module:} We monitor and preprocess the tweet stream continuously, and we utilize a fast filter module to obtain more possible relevant tweets for each profile. In order to decrease the time delay, as soon as the system gets a possible relevant tweet, the system will immediately estimate the relevance between expanded query and the tweet. We will update the query-biased relevance threshold every day instead of using a time window.
If the system judges a tweet as relevant tweet of a interest profile, the module of Novelty Verification will utilize a greedy clustering algorithm to decide whether the tweet can be regarded a new cluster compared with previous pushed Tweet pool. Once the tweet is regarded as novel tweet, the tweet will be pushed and appended into the pushed Tweet pool.
\end{itemize}

\subsection{Fast Filter}
In order to boost the speed of identifying possible relevant tweets for each profile, 
we simply filter tweets that do not contain any keywords for each profile,
and the rest tweets are chosen as candidate tweet collection.

\subsection{Relevance Estimation}
We utilize the KL-divergence language model based retrieval method to measure the relevance between query language model $\widehat{\theta}_Q$ and tweet language model $\widehat{\theta}_T$. The smoothing methods we use for language model are: (a) DIR (Bayesian Smoothing with Dirichlet Priors) smoothing, (b) JM (Jelinek-Mercer method) smoothing.
\begin{equation}
\label{equ:lm}
LMScore(T,Q) = \sum_{w \in Q} P(w|\widehat{\theta}_Q) \cdot logP(w|\widehat{\theta}_T)
\end{equation}

We incorporate the two scores using different smooth methods by linear interpolation.
\begin{equation}
\begin{aligned}
\label{equ:merge}
Score(T,Q) = \lambda \cdot LMScore_{Dir}(T,Q) \\+ (1-\lambda) \cdot LMScore_{JM}(T,Q)
\end{aligned}
\end{equation}
Here we empirically set $\lambda$ as $0.5$.
\subsection{Adaptive Query-biased Filtering}
Considering the fact that different topics can affect different attentation to varying degrees,
thus the count of relevant tweets are quite distinct.
Thus we utilize two strategies to estimate the query-biased relevance threshold $\beta$, namely empirical setting and human assist selection. 
(1) empirical setting method tries to utilize the popularity and relevance threshold in previous day of each query. Taking advantage of the ranked tweet list in scenario B, in our experiments, we utilize the relevance score of the tweet ranked at top ten as the relevance threshold $\beta$ in scenario A of next day. 
(2) Human assist selection also utilizes the ranked tweet list in scenario B, while here human will involve and quickly scan the top 100 tweets (we think top 100 is enough) and decide which one's score is the lower bound. In other words, we will quickly scan the ranked list from top to bottom, once we find one tweet is not relevant, we choose the relevance score of the tweet as the relevance threshold $\beta$ of the query in the next day.

Since the pushed tweets in the total evaluation period should be non-redundant,
we adaptively update the threshold of each interest profile.
Here we maintain a pushed tweet pool and utilize a greedy algorithm to determine whether a coming relevant tweet is novel or not \cite{fei2015handling,albakour2013sparsity}. We will calculate the relevance score between coming tweet and all the pushed tweets using the language model described in Relevance Estimation, then we greedily choose the pushed tweet that has highest relevance score with the coming tweet as reference. Once the highest relevance score is less than empirical novel threshold $\gamma$, we regard the coming tweet as novel tweet and push the tweet. Finally we append the coming tweet into the pushed tweet pool for estimating novelty of subsequent tweets.

\section{Scenario B: Email Digest}
In the email digest scenario,
we will identify a batch of up to 100 ranked tweets per day per interest profile.
At a high level, these results should be relevant and novel.
Timeliness is not important as long as the tweets were all posted on the previous day.

As shown in Fig.\ref{fig:Bsys}, our system for this scenario mainly contains four modules:

\begin{figure}[htbp]
\centering{
	\epsfig{file=figures/b.pdf,width=0.4\textwidth}
}
\caption{The System Architecture of the Email Digest Scenario.}
\label{fig:Bsys}
\end{figure}

\begin{itemize}
\item \textbf{Data Cleaning Module:}
We preprocess all tweets during evaluation period.
And we simply filter tweets that do not contain any keywords for each interest profile,
and the rest tweets are chosen as candidate tweet collection,
which will accelerate identifying possible relevant tweets for each profile.

\item \textbf{Query Expansion Module:}
As microblog retrieval suffers severely from the vocabulary mismatch problem
(i.e. term overlap between query and tweet is relatively small).
To tackle this issue, we leverage web-based query expansion method to improve retrieval performance
\cite{zhai2011mbfb}.
As is known to all, Google search is the dominant search engine in the majority countries
over the world, which indexes billions\cite{arlington2008google} of web pages,
so that users can search for the information they desire through the use of keywords and operators.
Therefore, we take the interest profile as the keywords to search in Google with Google Search Engine API before the evalution period.
As the user interest profile offered by TREC 2016 are JSON-formatted structure and each profile includes four fields, topid, title, description and narrative.
Here we only use the topic keywords as our \emph{OriginQuery} since we utilize external web resource to depict the background information, noted as \emph{ExpansionQuery}.
We utilize the expanded query to represent the interest profile and then estimate the relevance between the query and tweets.

\item \textbf{Relevance Ranking Module:}
Similar with the push notifications scenario,
we utilize the text similarity function $f$ to measure the relevance between query and tweet.
Then, all the tweets are ranked based on their relevance score.

\item \textbf{Novelty Verfication Module:}
Once we obtain the ranked tweet list after relevance ranking,
we will traverse over them and judge novelty for each tweet,
until we collect enough tweets (the count of pushed tweets is up to $100$).
We use the text similarity function $g$ to measure the novelty between tweet
and the pushed tweets of interest profile.
When the novelty scroe is smaller than $\gamma$,
we think the tweet is a novel one for current interest profile and should be pushed.

In this module, there are two kinds of strategies to measure novelty between tweets:
(1) Negative KL-divergence. The higher relevance score between tweets, the less novelty they are.
(2) Simhash. It is a popular method to handle web page redundancy\cite{charikar2002similarity}.
Simhash is one where similiar items are hashed to similiar hash values
and we can calculate the bitwise hamming distance between hash values.
The closer hamming distance between two tweets is, the more similar they are.
The simhash code is calculated as follow,

\begin{equation}
\label{equ:lm}
Sim_{code} = sign(\sum_{i=1 \in n} w_{i} c_{i})
\end{equation}

where $w_{i}$ is the weight of term $i$ and $c_{i}$ is the hash code of term $i$, $sign$ is symbol function that make positive to 1 and negative to 0 for every bit in code.

\end{itemize}

\subsection{Parameter Selection}
$\gamma$ is tuned via grid search on TREC 2015 dataset,
which is showed in Table \ref{tab:paraB}.

\begin{table}[htbp]
\centering
\caption{Parameters of the Email Digest Scenario.}
\label{tab:paraB}
\begin{tabular}{lccc}
\hline
Run ID&f&g&$\gamma$\\
\hline
PKUICSTRunB1&JM($\lambda=0.2$)&DIR($\mu=100$)&0.73\\
PKUICSTRunB2&DIR($\mu=100$)&DIR($\mu=100$)&0.72\\
PKUICSTRunB3&DIR($\mu=100$)&SimHash&0.42\\
\hline
\end{tabular}
\end{table}



\section{Result and Analysis}
The evaluation of TREC 2016 Real-time Summarization track take place from Tuesday, August 2, 2016 00:00:00 UTC to August 11, 2016 23:59:59 UTC. And there are 203 interest profiles which participants will be responsible for tracking. During the evaluation period, participants must maintain a running system that continuously monitors the tweet sample stream.

For scenario A,  Table \ref{tab:resA} show the performance of our submitted three runs.
The primary evaluation metric for scenario A is EG-1. **** task A description.

\begin{table}[htbp]
\centering
\caption{Performance of submitted runs for scenario A}
\label{tab:resA}
\begin{tabular}{lrrrr}
\hline
Run ID&ELG&nCG\\
\hline
PKUICSTRunA1&0.1415&0.1566\\
PKUICSTRunA2&\textbf{0.3175}&\textbf{0.3127}\\
PKUICSTRunA3&0.1382&0.1711\\
\hline
\end{tabular}
\end{table}

Table \ref{tab:resB} reports our results for scenario B periodic email digest. The primary evaluation metric is nDCG1. As it turns out, PKUICSTRunB3 significantly outperforms both other runs, indicating that the Simhash method for novelty verification module is successful in identifying novel tweets. Both PKUICSTRunB1 and PKUICSTRunB2 adopt the KL-divergence language model, with DIR(Bayesian Smoothing with Dirichlet Priors) smoothing and JM(Jelinek-Mercer method) respectively. And the uniform novel thresholds are $\gamma=0.73$ and $\gamma=0.72$ training on the TREC 15 dataset. From Table \ref{tab:resB}, we can see that nDCG1 and nDCG0 are the same in PKUICSTRunB1 and PKUICSTRunB2, which means on each $"$silent day$"$, our system still pushed some tweets that are regarded as unrelated ones. Obviously, the thresholds do not fit well. Further investigation and experiments are needed to solve this issue.

\begin{table}[htbp]
\centering
\caption{Performance of submitted runs for scenario A}
\label{tab:resB}
\begin{tabular}{lrrr}
\hline
Run ID&nDCG1&nDCG0\\
\hline
PKUICSTRunB1&0.1423&0.1423\\
PKUICSTRunB2&0.1569&0.1569\\
PKUICSTRunB3&\textbf{0.2348}&0.0151\\
\hline
\end{tabular}
\end{table}

\section{Conclusion}
In this work, we present our systems for TREC 2016 Real-Time Summarization Track.
**********In the push notification on a mobile phone scenario, 
we apply an adaptive timely query-biased filtering framework which monitors and estimates the twitter stream with given interest profiles continuously and immediately.
In the email digest scenario,
We apply web-based query expansion using language model to rank candidate tweets and then we leverage two kinds of strategies to measure novelty between tweets. Experimental results show our effectiveness and efficiency of our system in both tasks.






\section{Acknowledgments}
The work reported in this paper is supported by the National Natural Science Foundation of China Grant 61370116.
%\section{Conclusion}
In this work, we present our systems for TREC 2016 Real-Time Summarization Track.
In the push notification scenario,
we pay main attention on designing a online system
for handling the real-time Twitter sample stream
and make proper push actions for each interest profile.
In the email digest scenario,
We apply web-based query expansion using language model to rank candidate tweets and
then we leverage two kinds of strategies to measure novelty between tweets.
Experimental results show our effectiveness and efficiency of our system in both tasks.




\bibliographystyle{aaai}
\bibliography{yelp}
\end{document}
