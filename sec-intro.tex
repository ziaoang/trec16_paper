\section{Introduction}
Information retrieval in microblogging environment has attracted increasing attention with the growing popularity of microblog. To boost the user experience in the real-time environment, the retrieval results should consist of non-redundant representative tweets during the evolution of a given topic. TREC first introduced Tweet Timeline Generation (TTG) task in 2014 \cite{TREC2014}. The putative user model is given as follows: ``I have an information need expressed by a query Q at time t and I would like a summary that captures relevant information.''. The system should detect and eliminate redundant tweets and then form a list of non-redundant, chronologically ordered tweets that occur before time $t$. Follow the idea of TTG, real-time filtering task is proposed in this year's Microblog track to explore technologies for monitoring a stream of social media posts with respect to user's interest profile. The task is composing of two scenarios, namely push notifications on a mobile phone and periodic email digest.

In the scenario of periodic email digest, we apply a language model framework to estimate the relevance between given interest profile with candidate tweets. 
In the query side, aside from the traditional pseudo relevance feedback based on top ranked tweets, external evidences from the Google search results are also utilized in our retrieval model to better understand the user intent. For each interest profile, we rank the candidate tweets of every day by integrating two retrieval scores which adopt two different smooth methods (i.e. Dirichlet Smooth and JM Smooth) through simple linear combination. We utilize a query-biased adaptive threshold $\beta$ to choose top $K$ tweets as smaller candidate collection $\Gamma$ to generate the email digest. For each interest profile $Q$, the initial threshold $\beta$ is set as the integrated score of the tenth relevant tweet from the previous day. Meanwhile, we update the threshold of each interest profile everyday with the same way from previous day. For each candidate tweet in collection $\Gamma$, we calculate the relevance scores between the candidate tweet with each tweet that has been pushed previously, a tuned novel threshold $\gamma$ is used to determine whether the candidate tweet is included in the email digest.

In the scenario of push notifications on a mobile phone, different from the other scenario, for each interest profile, when a candidate tweet comes, we immediately determine whether to push the tweet to the interest profile. We first estimate the relevance score between the tweet and interest profile using the normalized KL-divergence distance. For interest profile $Q$ in day $D$, we utilize the adaptive threshold $\beta$ in the previous scenario to decide whether the tweet is relevant to $Q$. Then we compare the tweet with previous pushed tweet in this scenario of $Q$. Similarly, we use a tuned novel threshold $\gamma$ to decide whether to push the tweet to the interest profile.

The rest of the paper is structured as follows: We first presents the preliminaries of both scenarios in Section 2. and then we introduce our approaches for periodic email digest scenario in Section 3. In Section 4, we describe our push notifications system in detail. Section 5 present our experimental evaluation. At last, we conclude the paper in Section 6.