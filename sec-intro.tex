\section{Introduction}
As with the microblog shch as Twitter rapidly getting popular, the information that microblog covered is rather numerous than expected. To explore user's interests and boost retrieval and recommendation performance in real-time environment, TREC first introduced real-time task in 2011. And the Real-Time Summarization Track in 2016 is a single real-time filtering task broken down into two scenarios, which is aiming to explore techniques for monitoring streams of social media posts with respect to users' interest profiles. Different from the last year's Microblog track, it requires an on-line decision, which means participating systems need to decide whether or not push notification for a tweet before seeing the subsequent tweets. And two scenarios are described as follow:

\begin{itemize}
\item \textbf{Scenario A (mobile notification):} Content that is identified as relevant and novel by a system based on the user's interest profile should be sent to the user in a timely fashion. 
\item \textbf{Scenario B (email digest):} Participating systems should identify tweets and aggregate them into an email digest. The email should be periodically sent to a user(e.g., nightly). Under that circumstances, users can read a longer story about the contents.
\end{itemize}


Remain TODO


In the scenario of push notifications on a mobile phone, different from the other scenario, for each interest profile, when a candidate tweet comes, we immediately determine whether to push the tweet to the interest profile. We first estimate the relevance score between the tweet and interest profile using the normalized KL-divergence distance. For interest profile $Q$ in day $D$, we utilize the adaptive threshold $\beta$ in the previous scenario to decide whether the tweet is relevant to $Q$. Then we compare the tweet with previous pushed tweet in this scenario of $Q$. Similarly, we use a tuned novel threshold $\gamma$ to decide whether to push the tweet to the interest profile.


In the scenario of email digest, similar with scenario A, we firstly clean the raw tweets generated between evaluation period. To better understand the user intent, we utilize Google web resource as external evidences to expand original query. Then, the language model framework is applyed to estimate the relevance between given interest profile with candidate tweets. For each interest profile, we rank the candidate tweets of every day by the relevance scores which adopt two different smooth methods. Once we obtain the ranked tweet list, we calculate the novelty scores between the candidate tweet with each tweet that has been pushed previously, the novelty threshold $\gamma$ is used to determine whether the candidate tweet is included in the email digest. And there are two kinds of strategies to measure the novelty, which will be explained in Section 4 in detail.


The remainder of the paper is organized as follows: We first introduce the preliminaries of both scenarios in Section 2. In Section 3 and Section 4, we describe our system for pushing notifications on a mobile phone task and periodic email digest task respectively. And then, we present our experimental results and analysis in Section 5. At last, we conclude our paper in Section 6.
